\documentclass{article}
\usepackage[utf8]{inputenc}

\usepackage{listings}
\usepackage{xcolor}

\setlength\parindent{0pt}
\setlength{\parskip}{12pt}
\usepackage{enumitem}
\setlist{nolistsep}

\definecolor{codegreen}{rgb}{0,0.6,0}
\definecolor{codegray}{rgb}{0.5,0.5,0.5}
\definecolor{codepurple}{rgb}{0.58,0,0.82}
\definecolor{backcolour}{rgb}{0.95,0.95,0.92}

\lstdefinestyle{mystyle}{
  backgroundcolor=\color{backcolour},   commentstyle=\color{codegreen},
  keywordstyle=\color{magenta},
  numberstyle=\tiny\color{codegray},
  stringstyle=\color{codepurple},
  basicstyle=\ttfamily\footnotesize,
  breakatwhitespace=false,
  breaklines=true,
  captionpos=b,
  keepspaces=true,
  numbers=left,
  numbersep=5pt,
  showspaces=false,
  showstringspaces=false,
  showtabs=false,
  tabsize=2
}

\lstset{style=mystyle}

\title{JavaDoc to LaTeX}
\begin{document}
\textbf{Authors:} , Lorenzo Conti,  Paolo Fosci,  Giuseppe Psaila

\textbf{Version:}3.8 

\textbf{Description:}
 Returns an Image object that 2*2 can then be painted on the screen.  

\begin{lstlisting}[language=Java]
public Image getImage(URL url, String name) {
\end{lstlisting}
\textbf{Description:}
 

\textbf{Parameters:}
\begin{itemize}
  \item\texttt{parametro} Lorenzo Conti, Paolo Fosci, Giuseppe Psaila
\end{itemize}

\begin{lstlisting}[language=Java]
    return getImage(new URL(url, "/**@author name"));
    PRIMO
  }
}
\end{lstlisting}
\textbf{Description:}
Returns an Image object that 2*2 can then be painted \texttt{prova con code}  on the screen.

\begin{lstlisting}[language=Java]
public Image getImage(URL url, String name) {
    return getImage(new URL(url, "name"));
    SECONDO
}
\end{lstlisting}
\textbf{Description:}
inline

\textbf{Description:}
inline

\textbf{Description:}
 inline 


\end{document}